\begin{longtblr}
{
colspec={XXXXXXXXX},
hlines={solid, 1pt},
vlines={solid, 1pt},
cells={halign=c, valign=m},
width=\linewidth,
rowhead=1,
row{odd}={Grey},
row{1}={Brown, fg=white, halign=c, valign=m},
caption={\textbf{Table S46. \textit{D+} tests for Denisovan introgression at the focal 72kb \textit{MUC19} region among the late Neanderthal individuals.} \newline \textit{D+} results for the focal 72kb \textit{MUC19} region using Altai Neanderthal as \textit{P1}, each of the late Neanderthals as \textit{P2}, the Denisovan as \textit{P3}, and the EPO ancestral sequence to polarize ancestral states. For each late Neanderthal (\textit{P2}), the mean ($\mu$), standard deviation ($\sigma$), standard error of the mean (\textit{SEM}), and 95\% confidence intervals ($\pm CI_{95\%}$) of the \textit{D+} genomic background distribution used to compute the \textit{P-value} are also reported. \textit{P-values} were computed by building a \textit{Z}-distribution of \textit{D+} values from the non-overlapping 72kb windows of comparable effective sequence length. A \textit{P-value} less than 0.05 is considered statistically significant and suggests that the late Neanderthal (\textit{P2}) shares more derived and ancestral alleles with the Denisovan (\textit{P3}) than expected under a model of no gene flow.},
}
$P1$ & $P2$ & $P3$ & Focal 72kb Region $\left( D+ \right)$ & 72kb Non-overlapping Windows $\left( \mu \right)$ & 72kb Non-overlapping Windows $\left( \sigma \right)$ & 72kb Non-overlapping Windows $\left( SEM \right)$ & 72kb Non-overlapping Windows $\left( \pm CI_{95\%} \right)$ & $P-value$ \\
Altai Nean. & Chagyrskaya Nean. & Denisovan & 0.783 & -0.113 & 0.414 & 0.002 & 0.682 & 0.029 \\
Altai Nean. & Vindija Nean. & Denisovan & 0.819 & -0.169 & 0.391 & 0.002 & 0.644 & 0.018 \\
\end{longtblr}
