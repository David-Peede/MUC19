\begin{longtblr}
{
colspec={XXXX},
hlines={solid, 1pt},
vlines={solid, 1pt},
cells={halign=c, valign=m},
width=\linewidth,
rowhead=1,
row{odd}={Grey},
row{1}={Brown, fg=white, halign=c, valign=m},
caption={\textbf{Table S9. $\boldsymbol{Q95_{AFR:B:DEN}(1\%, 100\%)}$ values for the focal 72kb and 742kb \textit{MUC19} regions among non-African populations in the 1000 Genomes Project.} \newline The $Q95_{AFR:B:DEN}(1\%, 100\%)$ values observed in the focal 72kb and 742kb regions for each non-African population, stratified by super population: Admixed Americans (AMR), South Asians (SAS), East Asians (EAS), and Europeans (EUR). The $Q95_{AFR:B:DEN}(1\%, 100\%)$ statistic quantifies $95^{th}$ percentile of the Denisovan alleles in the non-African population (\textit{B}) conditioned on: 1) the Denisovan allele is found in the homozygous state, and 2) the Denisovan allele is at low frequency ($<1\%$) in the African super population.},
}
Super Population & Population $\left( B \right)$ & Focal 72kb Region ($Q95_{AFR:B:DEN}$ $(1\%, 100\%)$) & Focal 742kb Region ($Q95_{AFR:B:DEN}$ $(1\%, 100\%)$) \\
AMR & MXL & 0.305 & 0.305 \\
AMR & PEL & 0.218 & 0.218 \\
AMR & CLM & 0.074 & 0.074 \\
AMR & PUR & 0.087 & 0.087 \\
SAS & BEB & 0.157 & 0.157 \\
SAS & STU & 0.113 & 0.113 \\
SAS & ITU & 0.108 & 0.108 \\
SAS & PJL & 0.052 & 0.052 \\
SAS & GIH & 0.097 & 0.097 \\
EAS & CHB & 0.034 & 0.034 \\
EAS & KHV & 0.091 & 0.091 \\
EAS & CHS & 0.04 & 0.038 \\
EAS & JPT & 0.029 & 0.029 \\
EAS & CDX & 0.124 & 0.124 \\
EUR & TSI & 0.051 & 0.051 \\
EUR & CEU & 0.015 & 0.015 \\
EUR & IBS & 0.033 & 0.033 \\
EUR & GBR & 0.005 & 0.005 \\
EUR & FIN & 0.005 & 0.005 \\
\end{longtblr}
