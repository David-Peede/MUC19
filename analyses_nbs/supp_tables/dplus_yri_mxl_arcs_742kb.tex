\begin{longtblr}
{
colspec={XXXXXXXXX},
hlines={solid, 1pt},
vlines={solid, 1pt},
cells={halign=c, valign=m},
width=\linewidth,
rowhead=1,
row{odd}={Grey},
row{1}={Brown, fg=white, halign=c, valign=m},
caption={\textbf{Table S42. \textit{D+} tests for introgression at the focal 742kb \textit{MUC19}.} \newline \textit{D+} results for the focal 742kb \textit{MUC19} region using the Yoruba in Ibadan, Nigeria population (YRI) as \textit{P1}, the introgressed haplotype in the focal MXL individual (i.e., NA19664) as \textit{P2}, the four high-coverage archaics as \textit{P3}, and the EPO ancestral sequence to polarize ancestral states. For each archaic(\textit{P3}), the mean ($\mu$), standard deviation ($\sigma$), standard error of the mean (\textit{SEM}), and 95\% confidence intervals ($\pm CI_{95\%}$) of the \textit{D+} genomic background distribution used to compute the \textit{P-value} are also reported. \textit{P-values} were computed by building a \textit{Z}-distribution of \textit{D+} values from the non-overlapping 742kb windows of comparable effective sequence length. A \textit{P-value} less than 0.05 is considered statistically significant and suggests that introgressed haplotype in MXL (\textit{P2}) shares more derived and ancestral alleles with the given archaic individual (\textit{P3}) than expected under a model of no gene flow.},
}
$P1$ & $P2$ & $P3$ & Focal 742kb Region $\left( D+ \right)$ & 742kb Non-overlapping Windows $\left( \mu \right)$ & 742kb Non-overlapping Windows $\left( \sigma \right)$ & 742kb Non-overlapping Windows $\left( SEM \right)$ & 742kb Non-overlapping Windows $\left( \pm CI_{95\%} \right)$ & $P-value$ \\
YRI & NA19664 & Denisovan & 0.377 & 0.002 & 0.072 & 0.001 & 0.119 & $9.889e^{-8}$ \\
YRI & NA19664 & Altai Nean. & 0.091 & 0.017 & 0.086 & 0.002 & 0.141 & 0.144 \\
YRI & NA19664 & Chagyrskaya Nean. & 0.381 & 0.018 & 0.088 & 0.002 & 0.145 & $7.375e^{-6}$ \\
YRI & NA19664 & Vindija Nean. & 0.383 & 0.019 & 0.088 & 0.002 & 0.145 & $7.505e^{-6}$ \\
\end{longtblr}
