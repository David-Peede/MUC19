\begin{longtblr}
{
colspec={p{2cm}XXXXp{2cm}p{2cm}p{2cm}p{2cm}p{1cm}p{1cm}X},
hlines={solid, 1pt},
vlines={solid, 1pt},
cells={halign=c, valign=m},
width=\linewidth,
rowhead=1,
row{odd}={Grey},
row{1}={Brown, fg=white, halign=c, valign=m},
caption={\textbf{Table S48. Sequence divergence between the \textit{Denisovan-like} haplotype in MXL and phased late Neanderthal haplotypes at the focal 72kb \textit{MUC19} region.} \newline The observed sequence divergence---i.e., the number of pairwise differences between a modern human haplotype and a late Neanderthal haplotype normalized by the effective sequence length---between the \textit{Denisovan-like} haplotype in the focal MXL individual (i.e., NA19664), who harbors two \textit{Denisovan-like} haplotypes and has no heterozygous sites at the focal 72kb region, and the phased haplotypes for both of the late Neanderthals. To assess the significance, we built a distribution of sequence divergence between each of the modern human's two chromosomes and the late Neanderthal's pseudo-haplotype, which was generated by randomly sampling one allele at each position from the unphased late Neanderthal's genotypes. For each late Neanderthal, the mean ($\mu$), standard deviation ($\sigma$), standard error of the mean (\textit{SEM}), and 95\% confidence intervals ($\pm CI_{95\%}$) of the pseudo-haplotype divergence genomic background distribution used to compute the \textit{P-value} are also reported. The \textit{P-value} represents the proportion of non-overlapping 72kb windows of comparable effective sequence length where the sequence divergence is less than or equal to that observed at the focal 72kb region. After correcting for four multiple comparisons---i.e., two per each modern human haplotype---a \textit{P-value} less than 0.0125 is considered statistically significant. Note that the \textit{Denisovan-like} in each of the late Neanderthal's corresponds to their second haplotype (i.e., Chagyrskaya Nean. Hap. 2 and Vindija Nean. Hap. 2). Also, note that at the phased 72kb region, the effective sequence length with respect to the NA19664 individual is 48119bp and 48444bp for the  Chagyrskaya and Vindija Neanderthal, respectively.},
}
Archaic Hap. & Focal 72kb Region (Pairwise Diffs. Hap. 1) & Focal 72kb Region (Pairwise Diffs. Hap. 2) & Focal 72kb Region (Seq. Div. Hap. 1) & Focal 72kb Region (Seq. Div. Hap. 2) & 72kb Non-overlapping Windows $\left( \mu \right)$ & 72kb Non-overlapping Windows $\left( \sigma \right)$ & 72kb Non-overlapping Windows $\left( SEM \right)$ & 72kb Nonoverlapping Windows $\left( \pm CI_{{95\%}} \right)$ & $P-value$ (Hap. 1) & $P-value$ (Hap. 2) \\
Chagyrskaya Nean. Hap. 1 & 168 & 168 & 0.003491 & 0.003491 & 0.001 & $6.078e^{-4}$ & $2.467e^{-6}$ & $4.836e^{-6}$ & 0.993 & 0.993 \\
Chagyrskaya Nean. Hap. 2 & 5 & 5 & 0.000104 & 0.000104 & 0.001 & $6.078e^{-4}$ & $2.467e^{-6}$ & $4.836e^{-6}$ & 0.003 & 0.003 \\
Vindija Nean. Hap. 1 & 169 & 169 & 0.003489 & 0.003489 & 0.001 & $6.046e^{-4}$ & $2.449e^{-6}$ & $4.801e^{-6}$ & 0.993 & 0.993 \\
Vindija Nean. Hap. 2 & 4 & 4 & 0.000083 & 0.000083 & 0.001 & $6.046e^{-4}$ & $2.449e^{-6}$ & $4.801e^{-6}$ & 0.002 & 0.002 \\
\end{longtblr}
