\begin{longtblr}
{
colspec={XXXXXX},
cells={halign=c, valign=m},
hlines={solid, 1pt},
vlines={solid, 1pt},
width=\linewidth,
rowhead=1,
row{odd}={Grey},
row{1}={Brown, fg=white, halign=c, valign=m},
caption={\textbf{Table S14. Frequency of introgressed tracts overlapping \textit{MUC19} short-read repeat region among 1000 Genomes Project super populations.} \newline The frequency of introgressed tracts---i.e., the number of introgressed tracts normalized by the total number of chromosomes---and the mean tract length stratified by super population: Admixed Americans (AMR), South Asians (SAS), East Asians (EAS), Europeans (EUR), and Africans (AFR). Introgressed tracts at the \textit{MUC19} short-read repeat region (hg19, Chr12:40876395-40885001) are significantly enriched in AMR individuals (Fisher's Exact Test, Odds Ratio: 5.940, \textit{P-value}: $1.953e^{-31}$), with AMR populations exhibiting a higher proportion of introgressed tracts compared to non-AMR populations, excluding AFR (Proportions \textit{Z}-Test, \textit{Z}-statistic: 13.269, \textit{P-value}: $1.742e^{-40}$). Note that a "---" denotes that there are no introgressed tracts overlapping short-read repeat region for that group.},
}
Super Population & Total Number of Chromosomes & Number of Introgressed Tracts & Introgressed Tract Frequency & Mean Tract Length \\
AMR & 694 & 109 & 0.157 & 547954.128 \\
SAS & 978 & 27 & 0.028 & 283666.667 \\
EAS & 1008 & 58 & 0.058 & 335327.586 \\
EUR & 1006 & 6 & 0.006 & 460500 \\
AFR & 1008 & 0 & 0 & --- \\
\end{longtblr}
